\documentclass{beamer}
\usetheme{Boadilla}
\usepackage[utf8]{inputenc}
\usepackage[czech]{babel}
\title{Kalibrace a monitorování astročásticových teleskopů}
\author{Daniel Staník}
\institute{
SLO UPOL}
\date{\today}

\setbeamertemplate{background} {\includegraphics[width=550 pt,height=\paperheight]{fast.png}}





\setbeamercolor{title}{fg=yellow}
\setbeamercolor{frametitle}{fg=yellow}
\setbeamercolor{normal text}{fg=green}
\setbeamercolor{caption name}{fg=yellow}

\begin{document}
\begin{frame}
\titlepage
\end{frame}



\begin{frame}
\frametitle{Úvod}
\begin{itemize}
 \item Detekce kosmických částic s extrémní energií (tzv. UHERC).
 \item Testování a úpravy pulzního UV kalibračního zdroje.
 \item Analýza kalibračních dat.
\end{itemize}

\end{frame}



\begin{frame}
\frametitle{Úvod}
\begin{itemize}
 \item Detekce kosmických částic s extrémní energií (tzv. UHERC).
 \item Testování a úpravy pulzního UV kalibračního zdroje.
 \item Analýza kalibračních dat.
\end{itemize}

\end{frame}



\begin{frame}
\frametitle{Detekce kosmických částic s extrémní energií}
\begin{itemize}
 \item energie $10^{18}$ až $10^{20}$ keV .
 \item Využití atmosféry jako kalorimetru. Vytvoření částicové a elektromagnetické spršky po zásahu energetickou částicí.
 \item Fluorescenční technika detekce - detekce deexcitačního slabého UV spektra.
\end{itemize}

\end{frame}





\begin{frame}
\frametitle{Teleskop FAST}
\begin{itemize}
 \item Fluorescenční teleskop pro detekci kosmických částic s extrémní energií (UHECR).
 \item Detekční část - čtyři fotonásobiče a superodrazná UV zrcadla.
 \item Dnes v provozu 4 prototypy.
 \item Budoucí účel - osazení velké plochy teleskopy tohoto typu a rekonstrukce spršek vyvolaných UHECR částicemi.
\end{itemize}

\end{frame}

\begin{frame}
\frametitle{Teleskop FAST}
\begin{columns}[t]
\column{.5\textwidth}
\centering
\begin{figure}[H]
\includegraphics[scale=0.3, angle = 0, origin = c]{../BachelorThesis/pictures/fastTheoretical.png}\\
\caption{Návrh teleskopu.}
\end{figure}
\column{.5\textwidth}
\centering
\begin{figure}[H]
\includegraphics[scale=0.09, angle = 0, origin = c]{../BachelorThesis/pictures/FASTReal}\\
\caption{Teleskop FAST.}
\end{figure}
\end{columns}



\end{frame}



\begin{frame}
\frametitle{Vývoj a testování kalibračního UV zdroje}
\begin{itemize}
 \item Nutnost kalibrace teleskopu jako optické soustavy. Detekční části díky různým vlivům podléhají degradaci.
 \item K tomu účelu - pulzní kalibrační UV zdroj založen na diodách. Netestován, nutnost ověření jeho funkčnosti a návrh případných úprav či jiných konceptů. K tomu účelu sestavena aparatura pro dlouhodobé měření stability zdroje.
 \item Nutno ověřit dva parametry - stabilita výkonu a stabilita geometrie pulzů. Měření prováděno v intervalu dvou týdnů. K měření výkonu užit PM16 měřící přístroj optického výkonu a k měření pulzní geometrie - fotonásobič + osciloskop.
\end{itemize}





\end{frame}

\begin{frame}
%\frametitle{Teleskop FAST}


 \begin{figure}[H]
 \centering
 \includegraphics[scale=0.07, angle = 270, origin = c]{../BachelorThesis/pictures/KarlsRuhe}
 \caption{Prototyp kalibračního zdroje.}
 \label{UVsource}
\end{figure}


\end{frame}





\begin{frame}
%\frametitle{Teleskop FAST}




 \begin{figure}[H]
 \centering
 \includegraphics[scale=0.07, angle = 0, origin = c]{../BachelorThesis/pictures/aprature1b}
 \caption{Testovací aparatura.}
 \label{UVsource}
\end{figure}


\end{frame}




\begin{frame}
\frametitle{Výsledky měření}
\begin{columns}[t]
\column{.5\textwidth}
\centering
\includegraphics[width=5cm,height=3.5cm]{../BachelorThesis/pictures/powers}\\
\includegraphics[width=5cm,height=4cm]{../BachelorThesis/pictures/Height}
\column{.5\textwidth}
\centering
\includegraphics[width=5cm,height=4cm]{../BachelorThesis/pictures/rise}\\
\includegraphics[width=5cm,height=4cm]{../BachelorThesis/pictures/Slope}
\end{columns}


\end{frame}




\begin{frame}
\frametitle{Výsledky měření}
\begin{itemize}
 \item Nalezen zásadní problém - výrazný rostoucí trend ve výkonu. Potvrzeno fotonásobičem i PM16. Pulzní geometrie - doba náběhu a sklon nemají dlouhodobý trend.
 \item Hlavní příčinnou jsou degradační procesy v samotných diodách. Viz další stránka se samotným chováním diody.
 \item Možná oprava - přidaní zpětnovazební UV detekční diody, podle které se bude upravovat proud LEDkou. 

\end{itemize}


\end{frame}



\begin{frame}
\frametitle{Degradace osamocené LED diody.}
 \begin{figure}[H]
 \centering
 \includegraphics[scale=0.4, angle = 0, origin = c]{corrected1}
 \caption{Degradace osamocené LED diody.}
 \label{UVsource}
\end{figure}


\end{frame}



\begin{frame}
\frametitle{Optická zpětná vazba}


\begin{itemize}
 \item Nastavování výkonu podle detekované hodnoty - detekce výšky pulzů za pomoci fotodiody.
 \item Nutnost změření fotodiody - závislosti chování na teplotě, odezvy na pulzy a dlouhodobé stability.


\end{itemize}


 \begin{figure}[H]
 \centering
 \includegraphics[scale=0.045, angle = 0, origin = c]{../BachelorThesis/pictures/optomechanics.png}
 \caption{Zpětnovazební optomechanika.}
 \label{UVsource}
\end{figure}


\end{frame}


\begin{frame}

 \begin{figure}[H]
 \centering
 \includegraphics[scale=0.4, angle = 0, origin = c]{../BachelorThesis/pictures/1V}
 \caption{Závislost temných proudů na teplotě.}
 \label{UVsource}
\end{figure}


\end{frame}

\begin{frame}

 \begin{figure}[H]
 \centering
 \includegraphics[scale=0.4, angle = 0, origin = c]{../BachelorThesis/pictures/pulse.png}
 \caption{Detekovaný pulz přes I/U převodník.}
 \label{UVsource}
\end{figure}


\end{frame}






\begin{frame}

 \begin{figure}[H]
 \centering
 \includegraphics[scale=0.4, angle = 0, origin = c]{../BachelorThesis/pictures/ArtiAging.png}
 \caption{Stárnutí detekční fotodiody.}
 \label{UVsource}
\end{figure}


\end{frame}



\begin{frame}

\begin{itemize}
 \item 
 Použití desky Nucleo F446RE. Nastavení synchronního vzorkování za pomoci provázání interních časovačů.
 \item 
 Navzorkování pulzů, a vyvození výšky pulzu - získání hodnoty aktuálního výkonu. Tato hodnota následně použita do PID regulátoru. 
 
 \end{itemize}
 \begin{figure}[H]
 \centering
 \includegraphics[scale=0.3, angle = 0, origin = c]{../BachelorThesis/pictures/PWMSampling.png}
 \caption{Synchronizované vzorkování.}
 \label{UVsource}
\end{figure}


\end{frame}


\begin{frame}

 \begin{figure}[H]
 \centering
 \includegraphics[scale=0.4, angle = 0, origin = c]{../BachelorThesis/pictures/LongTime.png}
 \caption{Test upraveného zdroje.}
 \label{UVsource}
\end{figure}


\end{frame}



\begin{frame}
\frametitle{Analýza kalibračních dat}
\begin{itemize}
 \item Užití kalibračního UV zdroje umístěného do integrační koule, nasvěcování apertury teleskopu z různých poloh. 
 \item Hlavní účel analýzy - získání relativních odezvových konstant pro 4 fotonásobiče.
 \item Porovnání s teoretickými modely.
 
 
 
 \end{itemize}
 \end{frame}  
 
\begin{frame}
\frametitle{Nasvícení fotonásobičů}
 \begin{figure}[H]
 \centering
 \includegraphics[scale=0.265, angle = 0, origin = c]{../BachelorThesis/pictures/CalibPulses}
 \caption{Signál viděný fotonásobiči.}
 \label{UVsource}
\end{figure}


\end{frame}  
 
\begin{frame}
\frametitle{Ukázka distribuce}
 \begin{figure}[H]
 \centering
 \includegraphics[scale=0.18, angle = 0, origin = c]{../BachelorThesis/pictures/right}
 \caption{Ukázka fitování distribucí maximální výšky kalibračních pulzů a srovnání pro 4 fotonásobiče.}
 \label{UVsource}
\end{figure}


\end{frame} 


\begin{frame}
\frametitle{Porovnání výsledků se simulacemi}

\begin{table}[H]
\centering
\begin{tabular}{|c|c|c|c|}
\hline
   & pravá & levá & spodek \\ \hline
$c_0$ & $0.2314 \pm 0.0002$    & $1$   				   & $0.1070 \pm 0.0002$     \\ \hline
$c_1$ & $0.1397 \pm 0.0002$    & $0.7485 \pm 0.0003$   & $0.8061 \pm 0.0003$      \\ \hline
$c_2$ & $0.9304 \pm 0.0004$    & $0.2138 \pm 0.0002$   & $0.1631 \pm 0.0002$      \\ \hline
$c_3$ & $1$    				   & $0.2462 \pm 0.0002$   & $1$      \\ \hline
\end{tabular}
\caption{Naměřené odezvové konstanty.}
 \label{CalibConstTbl}
\end{table}






\begin{table}[H]

\begin{tabular}{|c|c|c|c|}
\hline
   & pravá & levá & spodek \\ \hline
$c_0$ & $0.48$    & $0.95$   & $0.44$     \\ \hline
$c_1$ & $0.41$    & $1$   	 & $0.98$      \\ \hline
$c_2$ & $1$    	  & $0.41$   & $0.46$      \\ \hline
$c_3$ & $0.95$    & $0.46$   & $1$      \\ \hline
\end{tabular}
\caption{Konstanty ze simulace.}
 \label{CalibConstTblSim}
\end{table}


\end{frame}





















\end{document}

